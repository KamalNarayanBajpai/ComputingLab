\documentclass[12pt]{article}
\usepackage[utf8]{inputenc}
\usepackage{graphicx}
\usepackage{amsmath, amssymb}
\usepackage{caption}
\usepackage{booktabs}
\usepackage{float}
\usepackage{geometry}
\geometry{margin=1in}

\title{Sample Research Paper}
\author{kamal Narayan Bajpai\\
	\small Department of CSE, NITK \\
	\small \texttt{kamal31narayanbajpai@gmail.com}
}
\date{\today}

\begin{document}
	\maketitle
	\begin{abstract}
		This is the file for the basic understanding of the tool latex.
	\end{abstract}
	
	\section{Introduction}
	
	we can write the content here about the headings 
	
	
	\section{Mathematical Equation writing }
	Here’s an inline equation: \( E = mc^3 \)
	
	And a displayed equation:
	\begin{equation}
		\int_0^\infty e^{-x^2} dx = \frac{\sqrt{\pi}}{2}
		\\label{1}
	\end{equation}
	
	\section{Tables}
	
	\subsection{Simple Table}
	
	\begin{table}[H]
		\centering
		\caption{Sample Data Table}
		\begin{tabular}{lcc}
			\toprule
			Item & Value A & Value B \\
			\midrule
			Item 1 & 10 & 20 \\
			Item 2 & 30 & 40 \\
			Item 3 & 50 & 60 \\
			\bottomrule
		\end{tabular}
	\end{table}
	
	\subsection{Nested Table (Table within Table)}
	
	\begin{table}[H]
		\centering
		\caption{Main Table with Nested Table}
		\begin{tabular}{|c|c|}
			\hline
			Main Category & Details \\
			\hline
			Category A &
			\begin{minipage}{0.5\textwidth}
				\centering
				\begin{tabular}{|c|c|}
					\hline
					Sub-item & Value \\
					\hline
					A1 & 5 \\
					A2 & 10 \\
					\hline
				\end{tabular}
			\end{minipage} \\
			\hline
			Category B & No nested data \\
			\hline
		\end{tabular}
	\end{table}
	
	\section{Images}
	
	\begin{figure}[H]
		\centering
		\includegraphics[width=0.6\textwidth]{example-image} % Replace with your image file
		\caption{Sample Image from LaTeX's default set}
	\end{figure}
	
	\section{Conclusion}
	
	This document demonstrates some of the core features used in writing academic papers with LaTeX.
	
\end{document}
